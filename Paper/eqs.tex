% \documentclass[10pt]{scrartcl}
\documentclass[10pt,twocolumn]{scrartcl}

\usepackage[utf8]{inputenc}
\usepackage[T1]{fontenc}
\usepackage[ngerman]{babel}

\usepackage{uarial}
\renewcommand{\familydefault}{\sfdefault}

\usepackage{amsmath}
\usepackage{amssymb}

\usepackage{graphicx}
\usepackage{tabularx}

\setlength{\parindent}{0cm}
\setlength{\parskip}{3mm}
\setlength{\textheight}{23.8cm}
\setlength{\headheight}{1cm}
\setlength{\topmargin}{-10mm}

\setlength{\oddsidemargin}{0cm}
\setlength{\evensidemargin}{0cm}
\setlength{\textwidth}{16cm}
\setlength{\columnsep}{8mm}

\usepackage{multicol}
\usepackage{colortbl}
\usepackage{xcolor}
\definecolor{grau}{gray}{0.95}
\definecolor{dunkelgrau}{gray}{0.85}

\usepackage[normal]{caption}
\usepackage{lipsum}

\setlength{\parindent}{5mm}
\setlength{\parskip}{0mm}

\usepackage{float}
\restylefloat{figure}

\renewcommand{\topfraction}{0.75}
\renewcommand{\textfraction}{0.2}

\newcommand{\abs}[1]{\ensuremath{\left\vert#1\right\vert}}

%###########################################################
% die Sachen mit der Kopfzeile
\usepackage{lastpage}
\usepackage{fancyhdr}
\fancyhf{} % leere alle Felder
\fancyhead[L]{\footnotesize Simulation des Keplerproblems} % Titel des Aufsatzes
\fancyhead[R]{\footnotesize Constantin Schneider, Nils Wende}
\fancyfoot[C]{\footnotesize \thepage/\pageref{LastPage}}
% \fancyfoot[C]{\footnotesize \thepage}
\renewcommand{\headrulewidth}{0.4pt} % obere Trennlinie
\pagestyle{fancy}
%###########################################################

\newcommand{\ownsection}[1]{\begin{center}\LARGE\bf#1\end{center}}

\begin{document}

\twocolumn[
\ownsection{Simulation des Keplerproblems}

\begin{center}
Constantin Schneider (constantinschneider94@gmail.com), Nils Wende (nils.wende@hotmail.com) \\
Mannheim, November 2015
\end{center}
\vspace*{5mm}
]

% \begin{multicols}{2}
\begin{equation}F = \abs{F_1} = \abs{F_2} = G \cdot \frac{m_1 \cdot m_2}{r^2}
\label{eq:newton}
\end{equation}

\begin{equation}x_{n+\frac{1}{2}}^i := x_n^i + \frac{1}{2} \cdot \Delta t \cdot v_n^i
\label{eq:22}
\end{equation}

\begin{equation}v_{n+1}^i := v_n^i + \Delta t \cdot a_{n+\frac{1}{2}}^i
\label{eq:23}
\end{equation}

\begin{equation}x_{n+1}^i := x_{n+\frac{1}{2}}^i + \frac{1}{2} \cdot \Delta t \cdot v_{n+1}^i
\label{eq:24}
\end{equation}

\begin{equation}x_{\frac{1}{2}}^i := x_0^i + \frac{1}{2} \cdot \Delta t \cdot v_0^i
+ \frac{1}{4} \cdot (\Delta t)^2 \cdot a_0^i
\label{eq:25}
\end{equation}

\begin{equation}F_x = F \cdot \cos \theta
\label{eq:2}
\end{equation}

\begin{equation}F_y = F \cdot \sin \theta
\label{eq:3}
\end{equation}

\begin{equation}\cos \theta = \frac{\Delta x}{r} = \frac{x_2 - x_1}{r}
\label{eq:4}
\end{equation}

\begin{equation}\sin \theta = \frac{\Delta y}{r} = \frac{y_2 - y_1}{r}
\label{eq:5}
\end{equation}

\begin{equation}F_x = G \cdot \frac{m_1 \cdot m_2 \cdot (x_2 - x_1)}{r^3}
\label{eq:8}
\end{equation}

\begin{equation}F = m \cdot a
\label{eq:9}
\end{equation}

\begin{equation}F_x = m_1 \cdot a_x
\label{eq:10}
\end{equation}

\begin{equation}
	\begin{aligned}
		F_x {} & = m_1 \cdot a_x \\
  			& = G \cdot \frac{m_1 \cdot m_2 \cdot (x_2 - x_1)}{r^3}
	\end{aligned}
\label{eq:11}
\end{equation}

\begin{equation}a_x = G \cdot \frac{m_2 \cdot (x_2 - x_1)}{r^3}
\label{eq:12}
\end{equation}

\begin{equation}
	\begin{aligned}
		r {} & = \sqrt{(\Delta x)^2 + (\Delta y)^2} \\
  			& = [(x_2-x_1)^2 + (y_2-y_1)^2]^\frac{1}{2}
	\end{aligned}
\label{eq:13}
\end{equation}

\begin{equation}a_x = G \cdot \frac{m_2 \cdot (x_2 - x_1)}
									{[(x_2-x_1)^2 + (y_2-y_1)^2]^\frac{3}{2}}
\label{eq:14}
\end{equation}

\begin{equation}a_y = G \cdot \frac{m_2 \cdot (y_2 - y_1)}
									{[(x_2-x_1)^2 + (y_2-y_1)^2]^\frac{3}{2}}
\label{eq:15}
\end{equation}

\begin{equation}a_x = G \cdot \frac{m_1 \cdot (x_1 - x_2)}
								{[(x_2-x_1)^2 + (y_2-y_1)^2]^\frac{3}{2}}
\label{eq:16}
\end{equation}

\begin{equation}a_y = G \cdot \frac{m_1 \cdot (y_1 - y_2)}
									{[(x_2-x_1)^2 + (y_2-y_1)^2]^\frac{3}{2}}
\label{eq:17}
\end{equation}

\begin{equation}a_{1_x} := G \cdot \frac{m_2 \cdot (x_2 - x_1)}
			{\alpha}
\label{eq:18}
\end{equation}

\begin{equation}a_{1_y} := G \cdot \frac{m_2 \cdot (y_2 - y_1)}
			{\alpha}
\label{eq:19}
\end{equation}

\begin{equation}a_{2_x} := G \cdot \frac{m_1 \cdot (x_1 - x_2)}
			{\alpha}
\label{eq:20}
\end{equation}

\begin{equation}a_{2_y} := G \cdot \frac{m_1 \cdot (y_1 - y_2)}
			{\alpha}
\label{eq:21}
\end{equation}

mit $\alpha := [(x_2-x_1)^2 + (y_2-y_1)^2]^\frac{3}{2}$ und $\alpha \neq 0$.

\begin{enumerate}
	\item Wähle einen kleinen Zeitschritt $\Delta t$.
	\item Wähle die Masse und komponentenweise die initialen Positions- und Geschwindigkeitswerte der Körper.
	\item Berechne die initialen Beschleunigungswerte $a_0^i$ mit den Gleichungen \eqref{eq:18} bis \eqref{eq:21}.
	\item Berechne mit den Werten der $a_0^i$ mit Gleichung \eqref{eq:25} die Werte von $x_{\frac{1}{2}}^i$.
	\item Wiederholung bis manueller Abbruch (Startwert $n:=0$):
	\begin{enumerate}
		\item Berechne mit den Werten der $x_{n+\frac{1}{2}}^i$ mit den Gleichungen \eqref{eq:18} bis \eqref{eq:21} die Werte der $a_{n+\frac{1}{2}}^i$.
		\item Berechne mit den Werten der $a_{n+\frac{1}{2}}^i$ mit Gleichung \eqref{eq:23} die Werte der $v_{n+1}^i$.
		\item Berechne mit den Werten der $v_{n+1}^i$ mit Gleichung \eqref{eq:24} die Werte der $x_{n+1}^i$.
		\item Zeige die neuen Positionen der Massen an.
		\item Berechne mit den Werten der $x_{n+1}^i$ mit Gleichung \eqref{eq:22} die Werte der $x_{n+\frac{3}{2}}^i$.
		\item Setze $n := n+1$.
	\end{enumerate}
\end{enumerate}

\end{document}